\chapter*{Введение}							% Заголовок
\addcontentsline{toc}{chapter}{Введение}	% Добавляем его в оглавление

\todo{Нужен ли автореферат и как-то особо отформатированное введение?}

Cormen: \cite[wrewer]{Knuth} says:
Очередь с приоритетом~--- это структура данных, позволяющая поддерживать
множество элементов частично упорядоченным. Будем считать, что элемент~---
это пара $(\textit{key}, \textit{item})$, где $key$~--- это приоритет.
В данной работе рассматриаваются очереди с приоритетом, поддерживающие следующие
базовые операции: $\textbf{insert}(key, item)$, добавляющая элемент $(key, item)$
в множество; $\textbf{findMin()}$, возвращающая элемент из множества
с минимальным приоритетом; и $\textbf{extractMin()}$, вдобавок удаляющая
из множества элемент с минимальным приоритетом.

Данная структура данных хорошо изучена. В \todo{бородатом году Дональдом Кнутом
или типа того} \cite{Cormen} было предложено решение с помощью двоичной кучи, позволяющее выполнять
операции $\textbf{insert}$ и $\textbf{extractMin}$ за $O(\log n)$ сравнений, где $n$~---
количество элементов в множестве. В дальнейшем были предложены решения с лучшей
асимптотикой (в большинстве своём, однако, неприменимые на практике из-за высокой
скрытой константы); оптимальной с некоторой точки зрения является 
\emph{Brodal queue}, реализаующая \todo{не помню асимптотику}.

В данной работе будет изучена задача приоритетной очереди, оптимальной
при условии ограниченного числа удалений. Пусть операция \textbf{insert}
была вызвана $n$ раз, \textbf{extractMin} $k$ раз, при этом на каждую
операцию \textbf{insert} было затрачено $O(1)$ сравнений. Тогда известно
(и будет доказано), что на операции \textbf{extractMin} необходимо в худшем
случае потратить $O(k \log k)$ сравнений. Будет построена структура
данных, вплотную приближающаяся к данной нижней оценке (с точностью до
множителя $O(\log^* n)$).


% \input{common/characteristic} % Характеристика работы по структуре во введении и в автореферате не отличается (ГОСТ Р 7.0.11, пункты 5.3.1 и 9.2.1), потому её загружаем из одного и того же внешнего файла, предварительно задав форму выделения некоторым параметрам

% \textbf{Объем и структура работы.} Диссертация состоит из~введения, четырёх глав, заключения и~двух приложений.
%% на случай ошибок оставляю исходный кусок на месте, закомментированным
%Полный объём диссертации составляет  \ref*{TotPages}~страницу с~\totalfigures{}~рисунками и~\totaltables{}~таблицами. Список литературы содержит \total{citenum}~наименований.
%
% Полный объём диссертации составляет
% \formbytotal{TotPages}{страниц}{у}{ы}{}, включая
% \formbytotal{totalcount@figure}{рисун}{ок}{ка}{ков} и
% \formbytotal{totalcount@table}{таблиц}{у}{ы}{}.   Список литературы содержит  
% \formbytotal{citenum}{наименован}{ие}{ия}{ий}.
