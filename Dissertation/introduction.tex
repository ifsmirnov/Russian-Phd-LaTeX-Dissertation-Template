\chapter*{Введение}							% Заголовок
\addcontentsline{toc}{chapter}{Введение}	% Добавляем его в оглавление

Очередь с приоритетом "--- абстрактный тип данных, поддерживающий две обязательные
операции~--- добавить элемент и извлечь минимум.  Будем считать, что элемент "---
это пара $(\textit{key}, \textit{item})$, где $key$ "--- это приоритет.
В данной работе рассматриваются очереди с приоритетом, поддерживающие
базовые операции: $\textbf{insert}(key, item)$, добавляющая элемент $(key, item)$
в множество; и $\textbf{extractMin()}$, удаляющая и возвращающая
элемент с минимальным приоритетом.

Эта структура данных хорошо изучена. Ещё в прошлом веке
было предложено решение с помощью двоичной кучи\cite[с.~179]{Cormen}, позволяющее выполнять
обе операции за $O(\log n)$ сравнений, где $n$ "---
количество элементов в множестве. Позднее были предложены решения с лучшей
асимптотикой, обычно, однако, неприменимые на практике из-за высокой
скрытой константы. В 1996 году Gerth Brodal опубликовал структуру
\emph{Brodal queue}\cite{brodal}, реализующую вставку, нахождение минимума
и слияние двух куч за $O(1)$ и извлечение за $O(\log n)$ в худшем случае.
Эта оценка в некотором смысле теоретически оптимальна.

В данной работе будет изучена задача приоритетной очереди, оптимальной
при условии ограниченного числа удалений. Пусть операция \textbf{insert}
была вызвана $n$ раз, \textbf{extractMin} $k$ раз, при этом на каждую
операцию \textbf{insert} было затрачено $O(1)$ сравнений. Тогда известно
(и будет доказано), что на операции \textbf{extractMin} необходимо в худшем
случае потратить $O(k \log k)$ сравнений. Мы построим структуру данных,
приближающуюся к данной оценке с точностью до множителя $O(\log^* n)$.

В первой главе будет построена структура \SCH, иллюстрирующая общую идею
алгоритма. Для достижения показанного времени работы
мы использовали техники бутстрэппинга (<<кучи из куч>>) и буферизованных вставок.
Во второй главе на базе \SCH будут построены структуры
\CH[r] и \CH[*]. Оценки на количество сравнений указаны в таблице \ref{tab:compare}.

\begin{table}[b]
\centering
\parbox{15cm}{%
\caption{Сравнение асимптотик разных версий \CH} \label{tab:compare}%
}


\begin{tabular}{|c|c|c|}
\hline
Название & Сложность \textbf{insert} & Сложность \textbf{extractMin} \\
\hline
\SCH & $O(1)$ & $O(\log n)$ \\
\hline
\CH[r] & $O(r)$ & $O(\log^{(r)} n + r(\log k + \log r))$ \\
\hline
\CH[*] & $O(\log^* n)$ & $O(\log^* n(\log k + \log \log^* n))$ \\
\hline
\end{tabular}

\end{table}


% \input{common/characteristic} % Характеристика работы по структуре во введении и в автореферате не отличается (ГОСТ Р 7.0.11, пункты 5.3.1 и 9.2.1), потому её загружаем из одного и того же внешнего файла, предварительно задав форму выделения некоторым параметрам

% \textbf{Объем и структура работы.} Диссертация состоит из~введения, четырёх глав, заключения и~двух приложений.
%% на случай ошибок оставляю исходный кусок на месте, закомментированным
%Полный объём диссертации составляет  \ref*{TotPages}~страницу с~\totalfigures{}~рисунками и~\totaltables{}~таблицами. Список литературы содержит \total{citenum}~наименований.
%
% Полный объём диссертации составляет
% \formbytotal{TotPages}{страниц}{у}{ы}{}, включая
% \formbytotal{totalcount@figure}{рисун}{ок}{ка}{ков} и
% \formbytotal{totalcount@table}{таблиц}{у}{ы}{}.   Список литературы содержит  
% \formbytotal{citenum}{наименован}{ие}{ия}{ий}.
