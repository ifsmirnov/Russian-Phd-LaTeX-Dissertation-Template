\section{Асимптотика}
Докажем, что для любой последовательности из $n$ вызовов \textbf{insert}
и $k$ вызовов \textbf{extractMin} потребуется $O(n + k(\log n + \log k))$
сравнений.

% описание метода потенциалов
\todo{описать, что такое метод потенциалов -- надо ли?}

Воспользуемся методом потенциалов. Пусть $S_i$~--- состояние структуры
после выполнения $i$ операций, $c_i$~--- время выполнения $i$-й операции,
$\Phi(S_i)$~--- потенциальная функция, принимающая вещественные значения.

\begin{definition}
$\tilde{c_i} = c_i - \Phi(S_i) + \Phi(S_{i-1}), i > 0$~--- амортизированное время
выполнения $i$-й операции.
\end{definition}
Таким образом, амортизированное время выполнения операции~--- это реальное время
минус разность потенциалов, накопленная во время этой операции. Значит, суммарное
амортизированное время выполнения $N$ операций равно
$$
\sum_{i=1}^N\tilde{c_i} = \sum_{i=1}^N(c_i - \Phi(S_i) + \Phi(S_{i-1})) =
\sum_{i=1}^Nc_i - \Phi(S_N) + \Phi(S_0)
$$
Если $\Phi(S_N) - \Phi(S_0) \geq F(N)$, то величину $\sum_{i=1}^N\tilde{c_i} + F(n)$
можно использовать в качестве верхней оценки на суммарное время выполнения $N$ операций.

% Если определить потенциальную функцию таким образом, что $\Phi(S_N) \geq \Phi(S_0)$,
% то суммарное амортизированное время $\sum_{i=1}^N\tilde{c_i}$ будет нижней
% оценкой на суммарное реальное время выполнения $\sum_{i=1}^N c_i$. С другой стороны,

В качестве потенциала возьмём 
$$\Phi(S_i)=-( \text{ количество элементов в буфере}) \cdot C_b$$
для некоторой константы
$C_b$, которая будет определена позже.

\paragraph{Операция вставки}

\begin{theorem} \label{BufferIsLargeEnough}
Пусть $n$~--- количество вызовов операции \textbf{insert}, $b$~--- размер
буфера, вычисляющийся, как описано в пункте \ref{insert}.
Пусть также $n > 4$. \todo{эн больше чего?}
Тогда в любой момент времени $b - 3 < \log (n - b) < b - 1$.
\end{theorem}
\begin{proof}
остаётся читателю. \todo{Конечно же, нет.}
\end{proof}

\begin{itemize}
\item На добавление элемента в буфер тратится 0 сравнений.

$c_i = 0$, $\Delta\Phi = -C_b$, $\tilde{c_i} = C_b$.
\item Пусть требуется очистить буфер размера $b$. На построение двоичной кучи на $b$ элементах требуется
$O(b)$ сравнений \todo{(реф)}. Из теор. \ref{BufferIsLargeEnough}
следует, что высота MH не превосходит $b$, значит, на добавление в $MH$
также потребуется не более $C_C\cdot b$ сравнений для некоторой константы $C_c$.

$c_i \leq C_c \cdot b$, $\Delta\Phi = C_b \cdot b$, $\tilde{c_i} \leq b(C_c - C_b)$.
\end{itemize}

\paragraph{Операция поиска максимума}
Данная операция не изменяет структуру, следовательно, не влияет на потенциалы. Для того,
чтобы найти максимальный элемент, требуется выбрать максимум из буфера (потратив $O(b)$ сравнений)
и сравнить его с вершинами куч MH и HH (потратив $O(1)$ сравнений). Итого требуется $O(b) + O(1) = O(\log n)$
сравнений.

\paragraph{Операция удаления максимума}
Пусть максимум найден.
\begin{enumerate}
\item Максимум лежит в буфере. Сравнений для его удаления не требуется.

$c_i = 0$, $\Delta\Phi = C_b$, $\tilde{c_i} = -C_b$.
\item Максимум лежит в MH. Данная операция не затрагивает буфер, поэтому
потенциалы не изменяются. Пусть $H$~--- куча или элемент в вершине MH или HH, где был найден
максимум.
\end{enumerate}
