\chapter{Основная схема} \label{chapt-sch}

В этой главе будет построена структура данных \SCH, реализующая операции \textbf{insert}
за $O(1)$ и \textbf{extractMin} за $O(\log k + \log n)$\footnote{
    Вообще говоря, $k\leq n$, поэтому здесь можно написать $O(\log n)$.
Однако для общности с полученными в следующей главе асимптотиками
здесь будет использована именно такая запись.}
сравнений, где
за $n$ обозначено количество вставок, за $k$ "--- извлечений к моменту выполнения
очередной операции.
\section{Обзор}
\SCH будет построена на базе двоичной кучи. Для улучшения
асимптотики будет использоваться  буферизированная вставка и дополнительный буфер
ограниченного размера для извлечения элементов. Структура будет состоять из трёх
частей: буфер вставки размера $\log n$, в который добавляются элементы; промежуточной
<<кучи куч>> (двоичной кучи, элементами которой являются другие двоичные кучи)
\emph{Middle Heap} (\MH);
и конечной кучи (\emph{Head Heap}, \HH), буфера для извлечения.

Отношение порядка
на множестве непустых двоичных куч порождается отношением порядка на элементах,
лежащих в корне. То есть для сравнения двух куч нужно сравнить минимальные элементы
каждой из них.

При добавлении элемента он попадает в буфер вставки. Затем, если буфер переполняется,
на его элементах строится двоичная куча и добавляется в Middle Heap.

Для нахождения и удаления минимального элемента необходимо пробежаться по всем
элементам буфера, а также посмотреть в вершину Middle Heap и Head Heap.
После этого элемент нужно удалить из той кучи, в которой он найден. При этом
удаление минимального элемента из всех куч, кроме Head Heap, происходит следующим образом:
корень удаляется, а оба его поддерева добавляются в Head Heap, если они непусты.  
Таким образом размер Heap Heap поддерживается равным $O(k)$, и все операции, кроме
просмотра буфера, выполняются за $O(\log |\HH|)$ и не зависят от высоты других куч.

На добавление одного элемента требуется $O(1)$ времени амортизированно: накопив
буфер размера $\log n$, нужно потратить $O(\log n)$ действий на то, чтобы добавить
элемент в промежуточную кучу, что даёт требуемую оценку в среднем.

Можно видеть, что размер Head Heap в любой момент времени не превосходит $O(k)$, где
$k$ "--- количество удалений, поскольку глубина любой <<кучи куч>> константна.
Для удаления элемента необходимо: a) просмотреть буфер, b) посмотреть
на корень промежуточной кучи и Head Heap, c) выполнить $O(1)$ вставок в Head Heap.
Части (b) и (c) занимают $O(\log k)$ времени, часть (a) "--- $O(\log n)$.
Узким местом является добавление в буфер. Оказывается, что на буфере можно
рекурсивно построить аналогичную структуру, существенно уменьшив слагаемое
$O(\log n)$ в асимптотике удаления и не ухудшив при этом операцию вставки.

Далее в этой главе будет построена описанная структура данных,
проведена деамортизация и получены оценки в худшем случае.
В главе \ref{chapt-ch} будет построена улучшенная версия
структуры под названием \CH,
оценка на извлечение минимума будет улучшена до $O(\log^* n (\log k + \log \log^* n))$,
а также будут описаны возможные трейд-оффы между временем
добавления и извлечения минимума.

\section{Описание}

\begin{definition}
Пусть $\mathcal{X}$ "--- некоторое линейно упорядоченное множество.
Назовём $\mathcal{H}(\mathcal{X})$
множество $\{\mathcal{X} \cup \text{все непустые двоичные кучи над элементами }\mathcal{X}\}$.
Введём на $\mathcal{H}(\mathcal{X})$ линейный порядок следующим образом:
$x < y \xLeftrightarrow{def} repr(x) < repr(y)$, где
\[
repr(x) = \left\lbrace
    \begin{array}{ll}
    x,& x \in \mathcal{X} \\
    x.top(),& x\text{ "--- двоичная куча над элементами }\mathcal{X} \\
    \end{array}
\right.
\]
Иными словами, $repr(x)$~--- минимальный представитель объекта $x$.
\end{definition}
\begin{designation}
Назовём $\myH_k(\myX)$ множество $\underbrace{\myH(\myH(\dots(}_k\myX)\dots))$.
\end{designation}

Обозначим за $\myX$ множество элементов (то есть пар $(key, item)$),
над которыми оперирует структура данных. На нём определён линейный порядок.

\begin{definition}
$\lev(x) $--- \emph{уровень} элемента $x$, т.е. минимальное $k: x \in \myH_k(\myX),
x \notin \myH_{k-1}(\myX)$. Иными словами, это максимальная глубина вложенности
двоичных куч внутри элемента.
\end{definition}

Структура данных состоит из трёх частей:
\begin{enumerate}
\item \emph{буфер B} "--- массив элементов $\myX$ размера $b$. $b$ будет
изменяться в процессе работы алгоритма и поддерживаться примерно равным $\log n$
(формально ограничения на размер буфера описаны в алгоритме~\ref{algo-insert}).
В начале работы $b=1$.
Элементы изначально попадают в буфер.
\item \emph{промежуточная куча \MH (middle heap)} "--- двоичная куча над $\myH(\myX)\setminus\myX$.
При заполнении буфера на нём строится двоичная куча и добавляется в \MH.
\item \emph{конечная куча \HH  (Head Heap)} "--- двоичная куча над $\myH_2(\myX)$.
Её размер всегда равен $O(k)$.
\end{enumerate}

\subsection{Операция вставки} \label{insert}
\begin{algorithm}[h]
\DefineAlgoKeywords
 \KwData{$x \in \myX$}
 добавить $x$ в конец буфера\;
 \If{размер буфера $> \log_2 n$}{
     создать двоичную кучу $T$ из элементов буфера\;
     очистить буфер\;
     добавить кучу $T$ в \MH\;
 }
 \caption{Операция \textbf{insert}}
 \label{algo-insert}
\end{algorithm}

\subsection{Операция извлечения минимума} \label{findmin}
Добавим \MH в \HH. После этого максимальный элемент может находиться
либо в буфере, либо в вершине \HH. Просмотрим все элементы буфера
за $O(|B|)$ и вершину \HH. Если минимум найден в буфере, удалим его
и вернём. Иначе нужно удалить корень из \HH. 

Сложность в том, что вершины \HH "--- не элементы, а кучи. Опишем,
как корректно производить удаление в таком случае. Для этого введём
вспомогательную функцию \texttt{Yield}. На вход она принимает
произвольный аргумент из $\myH_2(\myX)$ и возвращает минимальный
элемент из $\myX$, содержащийся в множестве (по сути, корень корня корня...).
Каждая просмотренная вершина удаляется из своей кучи, её оба ребёнка
добавляются в \HH.

\begin{function}[h]
 \caption{Yield(x)}
 \eIf{$\text{x }\in \myX$}{
     \Return x;
 }{
     \lIf{x.hasLeftChild} {\HH.insert(x.leftChild)}
     \lIf{x.hasRightChild} {\HH.insert(x.rightChild)}
     \Return \Yield(x.top())\;
 }
\end{function}

\begin{algorithm}[h]
\DefineAlgoKeywords
 \KwData{\x "--- минимальный элемент или куча, содержащая его}
 \Switch{положение \x}{
  \uCase{буфер}{
      удалить \x из буфера и вернуть его\;
  }
  \Case{\HH}{
      T = \HH.top()\;
      \HH.extractMin()\;
      \Return \Yield(T)\;
  }
 }
 \caption{Операция \textbf{extractMin}}
 \label{algo-extractmin-simple}
\end{algorithm}

Теперь мы готовы описать алгоритм извлечения минимума (алгоритм \ref{algo-extractmin-simple}).
Для удаления минимального элемента его нужно найти, просмотрев буфер и корень \HH.
После этого нужно выполнить операцию \textbf{extractMin} из всех куч, содержащих элемент.
Причём <<честно>> эта операция выполняется только для верхнего уровня Head Heap; для
того, чтобы обработать остальные, вызывается процедура \Yield, добавляющая левое
и правое поддерево элемента в \HH вместо непосредственно его удаления. Это сделано
для того, чтобы асимптотика операции не зависела от высоты никакой кучи, кроме \HH.

\section{Деамортизация} \label{deamort-simple}
В данной части будет проведена деамортизация операции добавления, т.е. получена
оценка в $O(1)$ сравнений на добавление в худшем случае, и проведён анализ всех
операций.

\subsection{Куча с версиями}
Для деамортизации нам понадобится частично персистентная куча с поддержкой
отложенных операцией, которую мы назовём \emph{кучей с версиями}.
Неформально говоря, требуется делать следующее: добавлять элемент <<по чуть-чуть>>
так, чтобы он не был виден раньше времени, а потом атомарно <<переключиться>>,
чтобы добавленный элемент появился в куче. Кроме того, необходима возможность
прервать добавление в любой момент.

\begin{definition}
\emph{Куча с версиями} "--- надстройка над двоичной кучей,
поддерживающая следующие операции:
\begin{enumerate}
\item Удалить корень и вернуть два его поддерева. Если в данный момент происходит
    вставка, отменить её и также вернуть вставляемый элемент. После этой операции
    добавлений больше не будет.
\item Если в данный момент не происходит вставка, сделать элемент $x$
    текущим вставляемым элементом.
\item Если в данный момент происходит вставка, проделать $t$ операций по вставке.
\item Если вставка происходила и уже закончена, атомарно добавить вставленный элемент.
\end{enumerate}
\end{definition}

В каждой вершине двоичной кучи хранится два указателя на потомков "--- правого
и левого (возможно, пустые). Будем вместо каждого указателя хранить кортеж
пар (указатель, номер версии), где версия "--- некоторое натуральное число.
Кроме того, в корне делева будет отдельно храниться актуальный номер версии $V$,
изначально равный $0$.
Для того, чтобы получить явное дерево, нужно для каждой вершины взять
указатель с максимальной версией, не превосходящей $V$.

Для выполнения отложенной вставки нужно вставлять элемент как обычно, но вместо
изменения указателей создавать новые, версии $V+1$. После завершения вставки
можно атомарно перейти на новую версию, увеличив $V$. Если необходимо отменить
вставку и удалить корень, нужно просто вернуть оба поддерева корня (согласно версии
$V$) и удалённый элемент, а обоим детям корня установить версию равной $V$.

При добавлении нового указателя в вершину необходимо удалить из неё все указатели,
кроме актуального на данный момент. Несложно видеть, что при этом из каждой
вершины всегда будет исходить не более двух указателей, причём среди них
всегда будет актуальный.


\subsection{Деамортизация операции вставки}

Напомним, как проводится операция вставки: элемент добавляется в буфер,
а при заполнении буфера на нём строится двоичная куча и добавляется в
\MH. Теперь \MH будет кучей с версиями, и работу после переполнения буфера
можно разделить на три итерации:
\begin{enumerate}[label=\Roman*.]
\item Очистить буфер и построить двоичную кучу $T$ на элементах, которые в нём были
\item Отложенно добавить $T$ в \MH
\item Переключить версию в \MH, чтобы применить добавление
\end{enumerate}
Назовём эту процедуру \emph{балансировкой}.

Все эти операции будут равномерно выполнены, пока буфер заполняется в следующий
раз, требуя $O(1)$ дополнительной работы на каждое добавление элемента.
Это будет доказано в разделе \ref{ch:proof-simple}.

Для балансировки буфера размера $b$ нужно после каждого из последующих $b$
добавлений проделывать $C$ операций процедур I, II, III после добавления.
Таким образом, к моменту следующего переполнения буфера балансировка будет закончена.

\subsection{Удаления в процессе балансировки}

Во время балансировки структура находится в нестабильном состоянии. Если
в это время поступает запрос \textbf{extractMin}, необходимо отменить балансировку,
при этом не потеряв никакой информации. В этой главе будет описано, как это делать
в зависимости от того, во время какой стадии балансировки пришёл запрос.

Если балансировка находится в I стадии, необходимо достроить двоичную кучу на множестве
$T$, тем самым переведя балансировку в стадию II.  
Если балансировка находится в II стадии (в том числе после выполнения только что
описанной операции), необходимо вставить $T$ в \HH. После этого нужно в любом
случае вставить \MH в \HH.

После подготовки к удалению \MH оказывается пуста. Таким образом, для удаления
минимума достаточно просмотреть буфер и корень \HH так же,
как было описано в параграфе \ref{findmin}.

Теперь мы готовы целиком описать алгоритмы вставки в \SCH и удаления минимума.

\section{Алгоритм}
Алгоритмы \ref{algo-init-deamort}, \ref{algo-insert-deamort}, \ref{algo-extractmin-deamort}
описывают соответственно инициализацию структуры данных, добавление элемента и извлечение
минимального элемента.

\begin{algorithm}[t]
\DefineAlgoKeywords
\Begin{
    \BalState \Gets \NoAction \;
    \BufSize \Gets 0\;
    \ElemCount \Gets 0\;
    \C \Gets константа из леммы \ref{theo-balancing-constant} \;
}
\caption{Инициализация деамортизированной кучи}
\label{algo-init-deamort}
\end{algorithm}

\begin{algorithm}[p]
\DefineAlgoKeywords
 \KwData{$\x \in \myX$}
 \Begin{
     \If{$\BalState\ \in \{\StateI, \StateII\}$}{
        выполнить \C операций по балансировке\;
        \If{закончилась стадия I}{
            \BalState \Gets \StateII\;
        }
        \If{закончилась стадия II}{
            выполнить переключение версии \MH\;
            \BalState \Gets \NoAction\;
        }
     }
     добавить \x в конец буфера\;
     \BufSize \Gets \BufSize + 1\;
     \ElemCount \Gets \ElemCount + 1\;
     \If{$\BufSize > \log_2 \ElemCount$}{
        \BufSize \Gets 0\;
        \BalState \Gets \StateI\;
        начать балансировку на элементах буфера и очистить буфер\;
     }
 }
 \caption{Операция \textbf{insert} в деамортизированной куче}
 \label{algo-insert-deamort}
\end{algorithm}

\begin{algorithm}[p]
\DefineAlgoKeywords
 \Begin{
     \If{\BalState =\ \StateI}{
         закончить построение кучи на \T\;
         добавить \T в \HH\;
     }
     \If{\BalState =\ \StateII}{
         отменить отложенное добавление в \MH\;
         добавить \T в \HH\;
     }
     \lIf{\MH непуста}{
         вставить \MH в \HH
     }
     \BalState \Gets \NoAction\;
     просмотреть буфер B и вершину \HH, если куча непуста,
     найти среди них минимальный элемент\;
     \eIf{минимум в B}{
         удалить минимум из буфера\;
     }{
         T = \HH.top()\;
         \HH.extractMin()\;
         \Return \Yield(T)\;
     }
 }
 \caption{Операция \textbf{extractMin} в деамортизированной куче}
 \label{algo-extractmin-deamort}
\end{algorithm}

\newpage
\section{Доказательство корректности и асимптотики} \label{ch:proof-simple}
\begin{theorem}[о корректности и асимптотике] \label{theo-fast-correct}
Для любой последовательности из
$n$ операций \textbf{insert} и $k$ операций \textbf{extractMin},
в которой запрос \textbf{extractMin} может поступать только к непустой
куче, верно следующее:
\begin{enumerate}
\item (корректность) каждая операция \textbf{extractMin} удаляет минимальный элемент
из находящихся в куче к тому моменту;
\item (асимптотика) каждая операция \textbf{insert} требует $O(1)$ сравнений,
каждая операция \textbf{extractMin} требует $O(\log n + \log k)$ сравнений,
где $k$ "--- количество вызовов \textbf{extractMin} к тому моменту, причём
обе оценки верны в худшем случае.
\end{enumerate}
\end{theorem}

\bigskip

В дальнейшем в этом разделе символы $n$ и $k$ имеют указанное выше значение,
если явно не сказано иное.

\begin{lem} \label{theo-mh-size}
Пусть размер буфера при последнем переполнении был равен $b$. Тогда $|\MH| \leq 2^b$.
\end{lem}
\begin{proof}
Из условия на переполнение буфера из алгоритма \ref{algo-insert-deamort} имеем
$n-1 \leq 2^{b-1}$, откуда получаем $n \leq 2^b$. Но $|\MH| \leq n$, откуда
и следует требуемое неравенство.
\end{proof}

\begin{lem}\label{theo-balancing-constant}
Пусть размер буфера при последнем переполнении был равен $b$.
Тогда существует некоторая константа $C$ такая,
что балансировку можно выполнить за не более чем $b\cdot C$ сравнений в худшем случае.
\end{lem}
\begin{proof}
Балансировка состоит из двух частей: построение двоичной кучи на $b$ элементах
и добавление элемента в \MH. Первая часть реализуется за $O(b)$ сравнений
(\cite[с.~181]{Cormen}).
Вторая часть реализуется за $O(\log |\MH|)$ сравнений (\cite[с.~181]{Cormen}). Но из
теоремы \ref{theo-mh-size} мы имеем $\log_2 |\MH| \leq b$. Значит, балансировка
реализуется за $O(b)$ сравнений, и существование искомой константы $C$ следует
из определения <<$O$~большого>>.
\end{proof}

\begin{lem} \label{theo-single-bal}
Две балансировки не могут идти одновременно, т.е.
к моменту начала балансировки предыдущая балансировка уже завершилась.
\end{lem}
\begin{proof}
Заметим, что если переполнение буфера возникло при размере $b$, то следующее
переполнение возникнет при б\'ольшем размере буфера, поскольку логарифм "---
монотонная функция. Значит, в течение следующих $b$ запросов \textbf{insert}
балансировка не произойдёт. Во время каждого из этих запросов будет проведено
$C$ операций по балансировке, где $C$ "--- константа из леммы \ref{theo-balancing-constant}.
Но из этой же леммы известно, что $b \cdot C$ операций достаточно для завершения
балансировки. Значит, к моменту следующего переполнения балансировка будет завершена.
\end{proof}

\begin{lem} \label{theo-hk-top-is-min}
Пусть $H \in \myH_k$ для некоторого $k$ "--- <<куча куч \dots\ куч>> уровня $k$,
построенная на множестве элементов $X$. Тогда в корне $H$ находится минимальный элемент,
т.е. $\min\limits_{x\in X} x = H.\underbrace{\mathbf{top}().\dots.\mathbf{top}()}_{k \mathrm{ times}}$.
\end{lem}
\begin{proof}
Докажем индукцией по $\lev(H)$.

Если $\lev(H) = 0$, то $H$ "--- элемент $X$, и утверждение учевидно.

Пусть $\lev(H) = k > 0$. Тогда все элементы $H$ "--- кучи с уровнем $<k$,
и по индукции в их корнях лежит минимальное значение. Но по свойству
бинарной кучи в корне $H$ лежит минимальный из корней всех
элементов $H$, что и требовалось доказать.
\end{proof}

\begin{theorem}[о сохранении инвариантов] \label{theo-invariant}
В любой момент соблюдаются следующие инварианты:
\begin{enumerate}
\item \MH и все её элементы "--- корректные бинарные кучи;
\item \HH и все её элементы "--- корректные бинарные кучи;
\item Если балансировка находится в стадии II, то множество $T$, которое
вставляется в \MH "--- корректная бинарная куча;
\item $|\HH\;\!| \leq 6k$.
\end{enumerate}
\end{theorem}
\begin{proof}
При инициализации, когда структура пуста, все инварианты выполнены. Докажем,
что они выполняются после всех операций.

При выполнении операции \textbf{insert} происходит две вещи: непосредственно добавление
элемента в буфер и, возможно, несколько операций по балансировке. Добавление
в буфер не затрагивает ни одно из интересующих нас множеств. Посмотрим на балансировку.

При выполнении первой стадии балансировки, опять же, ни одно из интересующих
нас множеств не изменяется. К моменту завершения первой стадии множетство
$T$ представляет собой бинарную кучу согласно алгоритму \ref{algo-insert-deamort}
и в дальнейнем не изменяется до окончания балансировки. Значит, инвариант
(3) выполнен.

При выполнении второй стадии балансировки изменяется только <<будущая>>
версия \MH. К моменту завершения второй стадии <<будущая>> версия является
корректной кучей согласно алгоритму \ref{algo-insert-deamort}, кроме того,
вставляемый элемент является корректной кучей по инварианту (3). Значит,
инвариант (1) сохранится во время второй стадии балансировки и её завершения.

Операцию \textbf{extractMin} можно разбить на две части: завершение балансировки
и непосредственно извлечение минимального элемента. Провeдение первой стадии
балансировки не нарушает инварианты, как мы видели ранее. Для завершения
второй стадии необходимо добавить множества $T$ и \MH в \HH; оба этих множества
являются корректными кучами по инвариантам (1) и (3). Таким образом, мы не нарушим
инварант (2) и добавим в \HH не более двух элементов.

Если извлечение минимума произошло из буфера, инварианты не нарушаются. Если
извлечение произошло из \HH, то, исходя из операции \Yield, в \HH
добавилось не более $2 \cdot \lev \left(\HH.\mathbf{top}()\right)$ элементов. Кроме того,
все добавленные элементы "--- корректные бинарные кучи. Значит, инвариант (2)
выполнен.

$\lev \left(\HH.\mathbf{top}()\right) < \lev (\HH) \leq 3$. Значит, суммарно за одну операцию
\textbf{extractMin} в \HH добавляется не более $2 + 2\cdot 2 = 6$ элементов,
что доказывает сохранение инварианта (4).

\end{proof}

\begin{proof}[Доказательство теоремы \ref{theo-fast-correct} (о корректности и асимптотике)]
Напрямую следует из теоремы \ref{theo-invariant} о сохранении инвариантов и леммы
\ref{theo-hk-top-is-min}.
\end{proof}

