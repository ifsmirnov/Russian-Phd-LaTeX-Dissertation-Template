\chapter{Улучшение асимптотики} \label{chapt-ch}

В предыдущей главе была описана структура очереди с приоритетом, позволяющая
добавлять элемент за $O(1)$ сравнений и удалять минимум за $O(\log k + \log n)$
сравнений, где $n$ "--- количество добавлений, $k$ "--- количество удалений
к моменту вызова операции удаления. Обе оценки выполняются в худшем случае.

В этой главе вторая оценка будет улучшена.  Будет описана структура данных \CH,
позволяющая улучшить асимптотику удаления до $(O\log k + \log \log \dots \log n)$
для любого наперёд заданного константного числа итераций логарифма, сохранив при
этом константную сложность добавления. Кроме того, (напрямую из предыдущей)
будут получены очереди с приоритетом со сложностью вставки и удаления,
соответственно, ($O(r)$,~$O(\log^{(r)} n + r (\log k + \log r))$) для
любого $r > 0$ и ($O(\log^* n)$,~$O(\log^* n(\log k + \log \log^* n))$).

Посмотрим на операцию \textbf{extractMin} в структуре данных, описанной в предыдущей
главе. Видно, что узкое место в асимптотике "--- просмотр буфера и достраивание
кучи $T$, если приходится экстренно завершать балансировку на первой стадии:
эта часть работает за $O(\log n)$, в то время как остальные за $O(\log k)$.
Значит, нужно избавиться от полного просмотра буфера при удалении.

Вообще говоря, просматривать весь буфер не нужно, требуется только уметь быстро
находить минимум. Для этого можно рекурсивно построить такую же структуру данных
размера $\log n$ на элементах буфера и вместо просмотра буфера делать запрос к ней.
Во внутренней структуре, в свою очередь, будет свой буфер (размера $\log \log n$).
Появятся две промежуточных кучи MH: одна во внутренней структуре, <<куча куч>>,
вторая "--- во внешней, <<куча куч куч>>. Head Heap сохранится в единственном
экземпляре и будет использоваться как при внутренней балансировке, так и при внешней.

Основная идея \CH заключается в том, чтобы рекурсивно строить аналогичную
структуру на буфере предыдущего уровня до тех пор, пока буфер не станет достаточно
маленького размера. В зависимости от глубины вложенности может достигаться
разная комбинация асимптотик времени добавления и извлечения, как было сказано
в начале этой главы.

В первой части будет описана \CH глубины вложенности 2 с амортизированными оценками
временной сложности. В следующих частях этой главы будет более детально рассмотрена
произвольная глубина вложенности, произведена деамортизация и доказана асимптотика
и корректность.

\section{\CH вложенности 2}
Описываемая в этом разделе структура "--- надстройка над \SCH, в которой
для уменьшения размера буфера добавляется ещё один уровень буферизации.
Это позволяет уменьшить время добавления до $O(\log k + \log \log n)$.

\CH вложенности 2 (или \CH[2]) состоит из следующих частей:

\begin{enumerate}
\item Буфер, куда изначально попадают элементы. Его размер растёт при добавлении
    элементов и поддерживается примерно равным $\log \log n$.
\item Промежуточная куча \MH[1]. Сюда попадают кучи, построенные на элементах
    буфера. Размер \MH[1] не превосходит $O(\log n)$.
\item Промежуточная куча \MH[2], <<куча куч куч>>. Когда \MH[1] становится
    слишком большого размера, её нужно добавить в \MH[2].
\item Конечная куча \HH, используемая так же, как и в \SCH.
\end{enumerate}

Алгоритм~\ref{algo-insert-2} более подробно иллюстрирует вставку элемента.
В данном случае оценка $O(1)$ на вставку амортизированная, деамортизация будет
проведена позже.

\begin{algorithm}[h]
 \KwData{$x \in \myX$}
 добавить $x$ в конец буфера\;
 \If{размер буфера $> \log_2 \log_2 n$}{
     создать двоичную кучу $T$ из элементов буфера\;
     очистить буфер\;
     добавить кучу $T$ в \MH[1]\;
 }
 \If{размер \MH[1] $> \log_2 \log_2 n$}{
     добавить \MH[1] в \MH[2]\;
     очистить \MH[1]\;
 }
 \caption{Операция \textbf{insert} в \CH[2]}
 \label{algo-insert-2}
\end{algorithm}

Переполнение буфера возникает примерно каждые $\log\log n$ вставок
и требует $\log \log n$ дополнительных действий на вставку элемента
в \MH[1]. Переполнение
\MH[1] возникает примерно каждые $\log n$ вставок и требует $\log n$
действий на вставку в \MH[2]. Отсюда следует амортизированная оценка
в $O(1)$ сравнений на вставку одного элемента.

Для удаления необходимо, как и в \SCH, вставить \MH[2] и \MH[1] в \HH,
затем просмотреть буфер и вершину \HH и поступить согласно
алгоритму~\ref{algo-extractmin-simple}.

Описание \CH[2] приведено только для упрощения понимания того, как
устроена многоуровневая структура. В дальнейшем все описания и доказательства
будут приведены только для общего случая, т.~е.~\CH[r].

\section{\CH произвольной вложенности}
По аналогии с \CH[2] можно определить \CH[r] для произвольного натурального $r$.  
\CH[r] состоит из следующих частей:

\begin{enumerate}
\item Буфер, куда изначально попадают элементы. Его размер растёт при добавлении
    элементов и поддерживается примерно равным $\log^{(r)} n$.
\item Промежуточные кучи \MH[1], \dots, \MH[r]. Размер кучи \MH[t] не превосходит
    $\log^{(r-t)} n$, $1 \leq t < r$. Размер кучи \MH[r] не превосходит $n$.
\item Конечная куча \HH, используемая так же, как и в \SCH.
\end{enumerate}

Вставка происходит по аналогии с \SCH и \CH[2]: элемент добавляется в буфер,
при переполнении буфера на нём строится двоичная куча и вставляется в \MH[1],
при переполнении кучи \MH[1] она вставляется в \MH[2] и т.~д., то есть
когда очередная промежуточная куча (кроме максимальной) достигает своего предельного
размера, она вставляется в промежуточную кучу следующего уровня.
На каждом уровне при переполнении суммарно выполняется $O(n)$ действий,
поэтому суммарное амортизированное время вставки "--- $O(r)$.

Удаление минимума происходит по аналогии с предыдущими описанными структурами.
\MH[1], \dots, \MH[r] добавляются в \HH, затем минимум извлекается из \HH
или из буфера. Можно показать, что размер \HH не превосходит
$O(k \cdot r)$, поэтому на одно удаление требуется $O(\log k + \log r + \log^{(r)} n)$
сравнений. Обе заявленные асимптотики будут доказаны далее.

\section{\CH неограниченной вложенности}

В предыдущей секции количество промежуточных куч (параметр $r$) было фиксированным.
Размер каждой следующей кучи поддерживался равным экспоненте от размера
предыдущей. Можно сохранить это свойство, при этом не ограничивая количество куч.
Таким образом, \MH[1] переполняется при размере 2, \MH[2] "--- при размере $2^2 = 4$,
\MH[3] "--- при размере $2^{2^2} = 16$, \dots, \MH[t] "--- при размере ${}^{t}2$,
где ${}^{a}n$ "--- операция \emph{тетрации}.

\begin{definition}
Для любого положительного вещественного $a > 0$ и неотрицательного целого $n \geq 0$,
тетрацию ${}^na$ можно определить рекуррентно:
\begin{enumerate}
\item ${}^0a$ = 1,
\item ${}^na = a^{({}^{n-1}a)}$, $n > 0$.
\end{enumerate}
Иными словами, тетрация "--- это результат вычисления <<степенной башни>> высоты $n$
из чисел $a$~(\cite{tetration}).
\end{definition}

Одновременно можно избавиться от буфера и добавлять элементы сразу в \MH[1], поскольку
добавление в кучу размера $\leq 2$ требует $O(1)$ сравнений. В остальном
вставка элемента и извлечение минимума абсолютно аналогичны рассмотренным ранее структурам.

\section{Деамортизация}
Как и в случае с \SCH, деамортизация нужна только для процесса балансировки.
Здесь способ, рассмотренный в разделе~\ref{deamort-simple}, будет применён для
каждого уровня в отдельности. Вначале мы рассмотрим деамортизацию \CH[r],
затем описанный способ будет адаптирован для \CH[*].

\subsection{Деамортизация \CH[r]}
Сделаем все \MH[t] кучами с версией (см. параграф~\ref{heap-with-versions}).
При вставке возможны два вида переполнений:
\begin{enumerate}
\item Переполнение буфера. При таком переполнении нужно построить кучу на элементах
    буфера, а затем вставить её в \MH[1]. Назовём это \emph{балансировкой нулевого уровня}.
\item Переполнение одной из куч \MH[1], \dots, \MH[r-1]. При таком переполнении
    нужно только вставить переполнившуюся куча в промежуточную кучу следующего
    уровня. Назовём процесс вставки \MH[t] в \MH[t+1] \emph{балансировкой $t$-го уровня}.
\end{enumerate}

В любой момент времени на каждом уровне может происходить балансировка. При выполнении
операции \textbf{insert} надо провести несколько операций по балансировке на каждом из них.
Эти балансировки независимы: если какая-то куча \MH[t] переполнилась и вставляется
в \MH[t+1], её состояние <<замораживается>>, и в неё саму больше не будет ничего вставлено.
Если в момент переполнения \MH[t] в неё производится отложенная вставка,
то вставку нужно завершить и сделать вставляемый элемент единственным
элементом новой \MH[t].

Далее будет показано, что балансировка $t$-го уровня может выполняться
в течение $\log^{(r-t)}n$ вставок и требует $O(\log^{(r-t)}n)$ времени,
значит, деамортизацию на каждом уровне
можно выполнять за $O(1)$ сравнений при каждом вызове \textbf{insert}. Так можно
достичь оценки в $O(r)$ сравнений на операцию добавления в худшем случае.

Для того, чтобы выполнить операцию \textbf{extractMin}, нужно экстренно завершить
балансировку на каждом уровне, затем, как обычно, добавить \MH[1], \dots, \MH[r] в \HH
и извлечь минимум из \HH или из буфера. Завершение балансировки на первом уровне
требует достраивания кучи на элементах буфера и отмены вставки в \MH[1], для этого
необходимо $O(\log^{(r)} n)$ сравнений. Завершение балансировки на уровнях после
первого включает в себя только отмену вставки и производится за $O(1)$ (если быть
точным, за 0 сравнений).

\subsection{Деамортизация \CH[*]}
Деамортизация \CH[*] происходит абсолютно аналогично. Различие лишь в отсутствии
нулевого уровня  балансировки: в \CH[*], в отличие от \CH[r], нет буфера,
поэтому завершение балансировки происходит за $O(1)$ на каждом уровне.

\section{Алгоритм}
За $T_0$, \dots, $T_{r}$, \dots обозначены отложенно вставляемые кучи.
$T_t$ вставляется в \MH[t+1]. Во время первой стадии балансировки нулевого
уровня $T_0$ также обозначает кучу в процессе построения.

Инициализация и операции для \CH[r] показаны в
алгоритмах~\labelcref{alg:init-r,alg:insert-r,alg:extractmin-r}, для \CH[*] "---
в алгоритмах~\labelcref{alg:init-*,alg:insert-*,alg:extractmin-*}.
% ============== CascadeHeap[r] ===================

% ---------------- Initialization ---------------
\begin{algorithm}[p]
\DefineAlgoKeywords

\Begin{
    \BalState \Gets (\NoAction, \dots, \NoAction) ($r$ times)\;
    \BufSize \Gets 0\;
    \ElemCount \Gets 0\;
    \C \Gets константа из леммы \ref{th:bal-constant}\;
}
\caption{Инициализация \CH[r]}
\label{alg:init-r}
\end{algorithm}

% ---------------- Insertion -------------------
\begin{algorithm}[p]
\DefineAlgoKeywords

\KwData{$\x \in \myX$}
\Begin{
    \If{$\BalState{\textit{0}}\ \in \{\StateI, \StateII\}$}{
        выполнить \C операций по балансировке на нулевом уровне\;
        \If{закончилась стадия I}{
            \BalState{0} \Gets \StateII\;
        }
    }
    \For{$t = 0$ \KwTo $r-1$}{
        \If{$\BalState{\textit{t}} = \StateII$}{
            выполнить \C операций по балансировке на $t$-м уровне\;
            \If{закончилась стадия II}{
                выполнить переключение версии \MH[t+1]\;
                \BalState{t} \Gets \NoAction\;
            }
        }
    }
    добавить \x в конец буфера\;
    \BufSize \Gets \BufSize + 1\;
    \ElemCount \Gets \ElemCount + 1\;
    \If{$\BufSize > \log_2 \ElemCount$}{
        \BufSize \Gets 0\;
        \BalState{0} \Gets \StateI\;
        \T{0} \Gets буфер\;
        начать балансировку на элементах буфера и очистить буфер\;
    }
    \For{$t = 1$ \KwTo $r-1$}{
        \If{\MH[t].size() $> \log_2^{(r-t)} \ElemCount$}{
            \BalState{t} \Gets \StateII\;
            начать отложенно вставлять \MH[t] в \MH[t+1]\;
            \T{t} \Gets \MH[t]\;
            \tcp{\MH[t] теперь пуста}
            \If{\BalState{t} = \StateII}{
                положить \T{t-1} единственным элементом \MH[t]\;
            }
        }
    }
}
\caption{Операция \textbf{insert} в \CH[r]}
\label{alg:insert-r}
\end{algorithm}

% ---------------- Extraction -------------------
\begin{algorithm}[p]
\DefineAlgoKeywords
\Begin{
    \If{\BalState{0} =\ \StateI}{
        закончить построение кучи на \T{0}\;
        добавить \T{0} в \HH\;
        \BalState{0} \Gets \NoAction\;
    }
    \For{$t = 0$ \KwTo $r-1$}{
        \If{\BalState{\textit{t}} =\ \StateII}{
            отменить отложенное добавление в \MH[t+1]\;
            добавить \T{t} в \HH\;
            \BalState{t} \Gets \NoAction\;
        }
    }
    \For{$t = 1$ \KwTo $r$}{
        добавить \MH[t] в \HH\;
    }
    просмотреть буфер B и вершину \HH, если куча непуста,
    найти среди них минимальный элемент\;
    \eIf{минимум в B}{
        удалить минимум из буфера\;
    }{
        T = \HH.top()\;
        \HH.extractMin()\;
        \Return \Yield(T)\;
    }
}
\caption{Операция \textbf{extractMin} в \CH[r]}
\label{alg:extractmin-r}
\end{algorithm}


% ============== CascadeHeap* ===================


% ---------------- Initialization ---------------
\begin{algorithm}[p]
\DefineAlgoKeywords

\Begin{
    \BalState \Gets (\NoAction, \NoAction, \dots) \tcp*{infinite array}
    \MaxLevel \Gets 1\;
    \C \Gets константа из леммы \ref{th:bal-constant}\;
}
\caption{Инициализация \CH[*]}
\label{alg:init-*}
\end{algorithm}

% ---------------- Insertion -------------------
\begin{algorithm}[p]
\DefineAlgoKeywords
\SetKwFunction{max}{max}

\KwData{$\x \in \myX$}
\Begin{
    \For{$t = 1$ \KwTo \MaxLevel}{
        \If{$\BalState{\textit{t}} = \StateII$}{
            выполнить \C операций по балансировке на $t$-м уровне\;
            \If{закончилась стадия II}{
                выполнить переключение версии \MH[t+1]\;
                \BalState{t} \Gets \NoAction\;
            }
        }
    }
    добавить \x в \MH[1]\;
    \For{$t = 1$ \KwTo \MaxLevel}{
        \If{\MH[t].size() $\geq {}^t{2}$}{
            \BalState{t} \Gets \StateII\;
            начать отложенно вставлять \MH[t] в \MH[t+1]\;
            \T{t} \Gets \MH[t]\;
            \tcp{\MH[t] теперь пуста}
            \If{\BalState{t} = \StateII}{
                положить \T{t-1} единственным элементом \MH[t]\;
            }
            \MaxLevel \Gets \max{\MaxLevel, t+1}\;
        }
    }
}
\caption{Операция \textbf{insert} в \CH[*]}
\label{alg:insert-*}
\end{algorithm}

% ---------------- Extraction -------------------
\begin{algorithm}[p]
\DefineAlgoKeywords
\Begin{
    \If{\BalState{0} =\ \StateI}{
        закончить построение кучи на \T{0}\;
        добавить \T{0} в \HH\;
        \BalState{0} \Gets \NoAction\;
    }
    \For{$t = 0$ \KwTo $r-1$}{
        \If{\BalState{\textit{t}} =\ \StateII}{
            отменить отложенное добавление в \MH[t+1]\;
            добавить \T{t} в \HH\;
            \BalState{t} \Gets \NoAction\;
        }
    }
    \For{$t = 1$ \KwTo $r$}{
        добавить \MH[t] в \HH\;
    }
    просмотреть буфер B и вершину \HH, если куча непуста,
    найти среди них минимальный элемент\;
    \eIf{минимум в B}{
        удалить минимум из буфера\;
    }{
        T = \HH.top()\;
        \HH.extractMin()\;
        \Return \Yield(T)\;
    }
}
\caption{Операция \textbf{extractMin} в \CH[*]}
\label{alg:extractmin-*}
\end{algorithm}


\newpage
\section{Доказательства} \label{sec:proof}
\begin{theorem}[о корректности и асимптотике {\CH[r]}] \label{th:fast-correct-r}
Для любого натурального $r > 1$, для любой последовательности из
$n$ операций \textbf{insert} и $k$ операций \textbf{extractMin},
в которой запрос \textbf{extractMin} может поступать только к непустой
куче, верно следующее:
\begin{enumerate}
\item (корректность) каждая операция \textbf{extractMin} удаляет минимальный элемент
из находящихся в куче к тому моменту;
\item (асимптотика) каждая операция \textbf{insert} требует $O(r)$ сравнений,
каждая операция \textbf{extractMin} требует $O(\log^{(r)} n + r(\log k + \log r))$ сравнений,
где $k$ "--- количество вызовов \textbf{extractMin} к тому моменту, причём
обе оценки верны в худшем случае.
\end{enumerate}
\end{theorem}

\begin{theorem}[о корректности и асимптотике {\CH[*]}] \label{th:fast-correct-*}
Для любой последовательности из
$n$ операций \textbf{insert} и $k$ операций \textbf{extractMin},
в которой запрос \textbf{extractMin} может поступать только к непустой
куче, верно следующее:
\begin{enumerate}
\item (корректность) каждая операция \textbf{extractMin} удаляет минимальный элемент
из находящихся в куче к тому моменту;
\item (асимптотика) каждая операция \textbf{insert} требует $O(\log^* n)$ сравнений,
каждая операция \textbf{extractMin} требует $O(\log^* n(\log k + \log \log^* n))$ сравнений,
где $k$ "--- количество вызовов \textbf{extractMin} к тому моменту, причём
обе оценки верны в худшем случае.
\end{enumerate}
\end{theorem}

\bigskip

Как и в секции \ref{ch:proof-simple}, для доказательства этих двух теорем
будет сформулировано и доказано несколько лемм. Если явно не сказано иное,
каждая лемма относится как и к \CH[r], так и к \CH[*]. Все леммы и теоремы,
в которых фигурирует буфер, относятся только к \CH[r]. Во всех доказательствах
параметр $r$ считается произвольным положительным натуральным числом.

\begin{remark}
Если явно не сказано иное, под записью $\log x$ подразумевается
двоичный логарифм ($\log_2 x$). То же самое относится к повторному
логарифму ($\log^{(r)} x$) и к итеративному ($\log^* x$).
\end{remark}

\begin{lem} \label{th:mh-few-elements}
Пусть в \MH[t] для некоторого $t$ переполнилась в некоторый
момент времени. Тогда для любого $b \geq 0$ после $b$
вставок $|\MH[t]| \leq b + 1$.
\end{lem}
\begin{proof}
Сначала заметим, что операция \textbf{extractMin}
может только уменьшить
количество элементов в \MH[t].

Элементы могут попасть в \MH[t] двумя способами:
\begin{enumerate}
\item в результате переполнения \MH[t] в неё попадает
куча, отложенно вставляемая в \MH[t];
\item в \MH[t] добавляется новый элемент в результате
завершившейся балансировки на уровне $t-1$.
\end{enumerate}

После события первого типа $|\MH[t]| = 1$. Событие второго
типа может произойти не чаще, чем один раз на вставку, потому
что две балансировки не могут идти одновременно.
Значит, при увеличении $b$ на $1$
в \MH[t] попадает не более одного элемента, что и доказывает
лемму.
\end{proof}

\begin{lem} \label{th:mh-exp-growth}
Пусть на $t$-м уровне ($t \geq 0$, в случае \CH[*] $t > 0$)
произошло переполнение кучи \MH[t] или буфера
после $n$ вставок, и следующее переполнение на этом уровне
произойдёт после $n' = n + b$ вставок для некоторого $b$,
если за это время не произойдёт вызова \textbf{extractMin}.
Тогда $|\MH[t+1]| \leq 2^b$.
\end{lem}
\begin{proof}[Доказательство для {\CH[r]}]\belowdisplayskip=-14pt
Обозначим $s = \log^{(r-t+1)} n$. Из условий на переполнение имеем:
\begin{gather*}
b > \log^{(r-t)} (n + b) \mathrm{\quad по\ лемме\ \ref{th:mh-few-elements}} \\
b > \log^{(r-t)} n = \log s \\
2^b > s \\
|\MH[t+1]| \leq \log^{(r+t-1)} n = s \mathrm{\quad из\ условий\ переполнения} \\
|\MH[t+1]| < 2^b
\end{gather*}
\end{proof}

\begin{proof}[Доказательство для {\CH[*]}]\belowdisplayskip=-14pt
\begin{gather*}
b \geq {}^t 2 \mathrm{\quad по\ лемме\ \ref{th:mh-few-elements}} \\
2^b \geq {}^{t+1} 2 \\
|\MH[t+1]| \leq {}^{t+1} 2\mathrm{\quad из\ условий\ переполнения} \\
|\MH[t+1]| \leq 2^b
\end{gather*}
\end{proof}


\begin{lem} \label{th:bal-constant}
Пусть на $t$-м уровне ($t \geq 0$, в случае \CH[*] $t > 0$)
произошло переполнение кучи \MH[t] или буфера
после $n$ вставок, и следующее переполнение на этом уровне
произойдёт после $n' = n + b$ вставок для некоторого $b$,
если за это время не произойдёт вызова \textbf{extractMin}.
Тогда найдётся константа $C$ такая, что балансировку можно
выполнить за не более чем $C \cdot b$ сравнений в худшем случае.
\end{lem}
\begin{proof}
Балансировка может состоять из не более чем двух стадий:
построение кучи на элементах буфера и вставка в промежуточную
кучу следующего уровня.

Рассмотрим первую стадию. Пусть размер буфера при переполнении
равен $s$. Тогда $s \leq b$ в силу монотонности повторного логарифма.
На $s$ элементах можно построить бинарную кучу за $O(s) \in O(b)$ сравнений
согласно~\cite[с.~181]{Cormen}.

Рассмотрим вторую стадию. Из леммы \ref{th:mh-exp-growth}
известно, что $|\MH[t]| \leq 2^b$. Вставку реализуется
за $O(|\MH[t]|)$ сравнений, что есть $O(b)$.

Обе части балансировки реализуются за $O(b)$ сравнений. Значит,
вся балансировка на отдельно взятом уровне реализуется за $O(b)$
сравнений, и существование искомой константы следует из определения
<<$O$~большого>>.
\end{proof}

\begin{lem} \label{th:one-balancing}
На каждом уровне одновременно не может идти более чем одна балансировка.
\end{lem}
\begin{proof}
Во время каждой операции \textbf{insert} на каждом уровне проделывается
$C$ шагов балансировки, где $C$~--- константа из леммы \ref{th:bal-constant}.
Но эта константа определена так, чтобы балансировка успела завершиться
за $b$ шагов, где $b$~--- минимально возможное количество вставок
до начала следующей балансировки.
\end{proof}

\begin{lem} \label{th:log*-levels}
Количество уровней в \CH[*] (т.~е. \textsf{MaximumLevel}) не превосходит
$\log^* n + 1$.
\end{lem}
\begin{proof}
Из условий на переполнение следует, что если $\mathsf{MaximumLevel} = t$,
то $n \geq {}^{t-1}2$. Значит, $\log^* n \geq t-1$, то есть
$t \leq \log^* n + 1$.
\end{proof}


\begin{lem} \label{th:small-mh-level}
Для любого валидного $t > 0$ верно
$\lev(\MH[t]) \leq t + 1$.
\end{lem}
\begin{proof}
Докажем индукцией по $t$.

Если $t=1$, то все элементы \MH[t]~--- кучи (уровня 1), построенные на элементах буфера
(в случае \CH[r]) или единичные элементы (в случае \CH[*]). В обоих случаях $\lev \MH[1] \leq 2$.

Если $t > 1$, то все элементы \MH[t] имеют уровень, не превосходящий максимальный уровень
\MH[t-1]. Тогда из предположения индукции получаем, что $\lev(\MH[t]) \leq \lev(\MH[t-1]) + 1 \leq t+1$.
\end{proof}


\begin{lem} \label{th:small-hh-level}
Положим
\[
R = \begin{cases}
r \mathrm{\ в\ случае\ \CH[r]} \\
\mathsf{MaximumLevel} \mathrm{\ в\ случае\ \CH[*]}
\end{cases}.
\]
Тогда $\lev(\HH) \leq R + 2$.
\end{lem}
\begin{proof}
Все элементы, попадающие в \HH, попадают туда из какой-то \MH[t]
или из кучи, которая когда-то была одной из \MH[t]. По построению
$t \leq R$. Из теоремы~\ref{th:small-mh-level} уровень добавляемого
элемента не превосходит $R+1$. Но тогда $\lev(\HH) \leq R+2$.
\end{proof}

\begin{theorem}[о сохранении инвариантов] \label{th:invar}
В любой момент сохраняются следующие инварианты:
\begin{enumerate}
\item \MH[t] и все её элементы~--- корректные бинарные кучи для любого валидного $t$;
\item \HH и все её элементы~--- корректные бинарные кучи;
\item Если балансировка на $t$-м уровне находится в стадии II, то множество \T{t}, которое
вставляется в \MH[t+1]~--- корректная бинарная куча;
\item[4r.] (версия {\CH[r]})\quad $|\HH| \leq (4r+4)k$;
\item[4*.] (версия {\CH[*]})\quad $|\HH| \leq k\cdot (4 \log^* n + 8)$.
\addtocounter{enumi}{-1}
\end{enumerate}
\end{theorem}
\begin{proof}
Доказательство этой теоремы техническое и построено так же, как и
доказательство теоремы~\ref{theo-invariant}.

При инициализации, когда структура пуста, все инварианты выполнены. Докажем,
что они выполняются после всех операций.

При выполнении операции \textbf{insert} происходит две вещи: непосредственно добавление
элемента в буфер или \MH[1] и, возможно, несколько операций по балансировке. Добавление
в буфер не затрагивает ни одно из интересующих нас множеств. Добавление в \MH[1]
происходит только в \CH[*]. Поскольку в ней нет нулевого уровня балансировки,
на инвариант (3) вставка не влияет. Инвариант (1) сохраняется по свойству вставки в кучу.

Посмотрим на балансировку.

При выполнении первой стадии балансировки, опять же, ни одно из интересующих
нас множеств не изменяется. К моменту завершения первой стадии множетство
\T{0} представляет собой бинарную кучу согласно алгоритму \ref{alg:insert-r}
и в дальнейнем не изменяется до окончания балансировки. Значит, инвариант
(3) выполнен.

При выполнении второй стадии балансировки изменяется только <<будущая>>
версия \MH. К моменту завершения второй стадии <<будущая>> версия является
корректной кучей согласно алгоритмам \ref{alg:insert-r} и \ref{alg:insert-*},
кроме того, вставляемый элемент является корректной кучей по инварианту (3). Значит,
инвариант (1) сохранится во время второй стадии балансировки и её завершения.

Операцию \textbf{extractMin} можно разбить на две части: завершение балансировки
и непосредственно извлечение минимального элемента. Провeдение первой стадии
балансировки не нарушает инварианты, как мы видели ранее. Для завершения
второй стадии необходимо добавить все множества \T{t} и \MH[t] в \HH; все они
являются корректными кучами по инвариантам (1) и (3). Таким образом, мы не нарушим
инварант (2).

Посчитаем, сколько элементов было добавлено в \HH. Для каждого уровня $t$ могло
быть добавлено не более двух новых элементов (\MH[t] и \T{t}).
Значит, в \CH[r] было добавлено не более $2r$ элементов, в \CH[*]~---
не более $2(\log^* n+1)$ (теор.~\ref{th:log*-levels}).

Если извлечение минимума произошло из буфера, инварианты не нарушаются. Если
извлечение произошло из \HH, то, исходя из операции \Yield, в \HH
добавилось не более $2 \cdot \lev \left(\HH.\mathbf{top}()\right) \leq \lev(\HH)$ элементов. Кроме того,
все добавленные элементы "--- корректные бинарные кучи. Значит, инвариант (2)
выполнен.

В случае \CH[r] за одну операцию удаления в \HH было добавлено не более
$2r + 2\lev(\HH) \leq 2r + 2(r+2) = 4r+4$ элементов.

В случае \CH[*] в \HH было добавлено не более
$2(\log^*n + 1) + 2\lev(\HH) \leq 2(\log^*n+1) + 2(\log^*n+3)
= 4 \log^*n + 8$ элементов. Кроме того, $\log^*n$ монотонно неубывает
по $n$. Это доказывает соблюдение инвариантов (4r) и (4*).
В этих утверждениях мы использовали теорему~\ref{th:small-hh-level}.
\end{proof}

\begin{proof}[Доказательство теорем~\ref{th:fast-correct-r} и~\ref{th:fast-correct-*}]
Доказательство этих теорем напрямую следует из теоремы~\ref{th:invar}
о сохранении инвариантов и леммы~\ref{th:one-balancing}.
\end{proof}

Наконец, докажем теоретическую нижнюю оценку.

\begin{theorem} \label{th:lower_bound}
Пусть некоторая структура данных реализует операции \textbf{insert} и
\textbf{extractMin}, причём операция \textbf{insert} требует $O(1)$ сравнений
в худшем случае. Тогда для выполнения $k$ операций \textbf{extractMin}
требуется $\Omega(k \log k)$ сравнений в худшем случае.
\end{theorem}
\begin{proof}
Заметим, что выполнить $k$ извлечений минимума может оказаться не проще,
чем отсортировать $k$ элементов. Из теоретико-информационных соображений
известно, что на это требуется $\Theta(k \log k)$ сравнений\cite[с.~222]{Cormen}.

На все операции вставки было потрачено $O(n)$ сравнений.
Тогда при $k = \omega(\frac{n}{\log n})$ имеем $\Theta(k \log k) \notin O(n)$,
то есть сравнений, получаемых от добавления элементов, недостаточно
для проведения сортировки. Значит, в худшем случае
требуется $\Omega(k \log k)$ дополнительных сравнений.
\end{proof}

