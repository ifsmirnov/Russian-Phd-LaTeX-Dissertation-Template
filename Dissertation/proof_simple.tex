\begin{theorem}[о корректности и асимптотике] \label{theo-fast-correct}
Для любой последовательности из
$n$ операций \textbf{insert} и $k$ операций \textbf{extractMin},
в которой запрос \textbf{extractMin} может поступать только к непустой
куче, верно следующее:
\begin{enumerate}
\item (корректность) каждая операция \textbf{extractMin} удаляет минимальный элемент
из находящихся в куче к тому моменту;
\item (асимптотика) каждая операция \textbf{insert} требует $O(1)$ сравнений,
каждая операция \textbf{extractMin} требует $O(\log n + \log k)$ сравнений,
где $k$ "--- количество вызовов \textbf{extractMin} к тому моменту, причём
обе оценки верны в худшем случае.
\end{enumerate}
\end{theorem}

\bigskip

В дальнейшем в этом разделе символы $n$ и $k$ имеют указанное выше значение,
если явно не сказано иное.

\begin{lem} \label{theo-mh-size}
Пусть размер буфера при последнем переполнении был равен $b$. Тогда $|\MH| \leq 2^b$.
\end{lem}
\begin{proof}
Из условия на переполнение буфера из алгоритма \ref{algo-insert-deamort} имеем
$n-1 \leq 2^{b-1}$, откуда получаем $n \leq 2^b$. Но $|\MH| \leq n$, откуда
и следует требуемое неравенство.
\end{proof}

\begin{lem}\label{theo-balancing-constant}
Пусть размер буфера при последнем переполнении был равен $b$.
Тогда существует некоторая константа $C$ такая,
что балансировку можно выполнить за не более чем $b\cdot C$ сравнений в худшем случае.
\end{lem}
\begin{proof}
Балансировка состоит из двух частей: построение двоичной кучи на $b$ элементах
и добавление элемента в \MH. Первая часть реализуется за $O(b)$ сравнений
(\cite[с.~181]{Cormen}).
Вторая часть реализуется за $O(\log |\MH|)$ сравнений (\cite[с.~181]{Cormen}). Но из
теоремы \ref{theo-mh-size} мы имеем $\log_2 |\MH| \leq b$. Значит, балансировка
реализуется за $O(b)$ сравнений, и существование искомой константы $C$ следует
из определения <<$O$~большого>>.
\end{proof}

\begin{lem} \label{theo-single-bal}
Две балансировки не могут идти одновременно, т.е.
к моменту начала балансировки предыдущая балансировка уже завершилась.
\end{lem}
\begin{proof}
Заметим, что если переполнение буфера возникло при размере $b$, то следующее
переполнение возникнет при б\'ольшем размере буфера, поскольку логарифм "---
монотонная функция. Значит, в течение следующих $b$ запросов \textbf{insert}
балансировка не произойдёт. Во время каждого из этих запросов будет проведено
$C$ операций по балансировке, где $C$ "--- константа из леммы \ref{theo-balancing-constant}.
Но из этой же леммы известно, что $b \cdot C$ операций достаточно для завершения
балансировки. Значит, к моменту следующего переполнения балансировка будет завершена.
\end{proof}

\begin{lem} \label{theo-hk-top-is-min}
Пусть $H \in \myH_k$ для некоторого $k$ "--- <<куча куч \dots\ куч>> уровня $k$,
построенная на множестве элементов $X$. Тогда в корне $H$ находится минимальный элемент,
т.е. $\min\limits_{x\in X} x = H.\underbrace{\mathbf{top}().\dots.\mathbf{top}()}_{k \mathrm{ times}}$.
\end{lem}
\begin{proof}
Докажем индукцией по $\lev(H)$.

Если $\lev(H) = 0$, то $H$ "--- элемент $X$, и утверждение очевидно.

Пусть $\lev(H) = k > 0$. Тогда все элементы $H$ "--- кучи с уровнем $<k$,
и по индукции в их корнях лежит минимальное значение. Но по свойству
бинарной кучи в корне $H$ лежит минимальный из корней всех
элементов $H$, что и требовалось доказать.
\end{proof}

\begin{theorem}[о сохранении инвариантов] \label{theo-invariant}
В любой момент соблюдаются следующие инварианты:
\begin{enumerate}
\item \MH и все её элементы "--- корректные бинарные кучи;
\item \HH и все её элементы "--- корректные бинарные кучи;
\item Если балансировка находится в стадии II, то множество $T$, которое
вставляется в \MH "--- корректная бинарная куча;
\item $|\HH\;\!| \leq 6k$.
\end{enumerate}
\end{theorem}
\begin{proof}
При инициализации, когда структура пуста, все инварианты выполнены. Докажем,
что они выполняются после всех операций.

При выполнении операции \textbf{insert} происходит две вещи: непосредственно добавление
элемента в буфер и, возможно, несколько операций по балансировке. Добавление
в буфер не затрагивает ни одно из интересующих нас множеств. Посмотрим на балансировку.

При выполнении первой стадии балансировки, опять же, ни одно из интересующих
нас множеств не изменяется. К моменту завершения первой стадии множество
$T$ представляет собой бинарную кучу согласно алгоритму \ref{algo-insert-deamort}
и в дальнейшем не изменяется до окончания балансировки. Значит, инвариант
(3) выполнен.

При выполнении второй стадии балансировки изменяется только <<будущая>>
версия \MH. К моменту завершения второй стадии <<будущая>> версия является
корректной кучей согласно алгоритму \ref{algo-insert-deamort}, кроме того,
вставляемый элемент является корректной кучей по инварианту (3). Значит,
инвариант (1) сохранится во время второй стадии балансировки и её завершения.

Операцию \textbf{extractMin} можно разбить на две части: завершение балансировки
и непосредственно извлечение минимального элемента. Проведение первой стадии
балансировки не нарушает инварианты, как мы видели ранее. Для завершения
второй стадии необходимо добавить множества $T$ и \MH в \HH; оба этих множества
являются корректными кучами по инвариантам (1) и (3). Таким образом, мы не нарушим
инварант (2) и добавим в \HH не более двух элементов.

Если извлечение минимума произошло из буфера, инварианты не нарушаются. Если
извлечение произошло из \HH, то, исходя из операции \Yield, в \HH
добавилось не более $2 \cdot \lev \left(\HH.\mathbf{top}()\right)$ элементов. Кроме того,
все добавленные элементы "--- корректные бинарные кучи. Значит, инвариант (2)
выполнен.

$\lev \left(\HH.\mathbf{top}()\right) < \lev (\HH) \leq 3$. Значит, суммарно за одну операцию
\textbf{extractMin} в \HH добавляется не более $2 + 2\cdot 2 = 6$ элементов,
что доказывает сохранение инварианта (4).

\end{proof}

\begin{proof}[Доказательство теоремы \ref{theo-fast-correct} (о корректности и асимптотике)]
Напрямую следует из теоремы \ref{theo-invariant} о сохранении инвариантов и леммы
\ref{theo-hk-top-is-min}.
\end{proof}
