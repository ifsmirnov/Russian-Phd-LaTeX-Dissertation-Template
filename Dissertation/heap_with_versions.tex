\subsection{Куча с версиями} \label{heap-with-versions}
Для деамортизации нам понадобится частично персистентная куча с поддержкой
отложенных операцией, которую мы назовём \emph{кучей с версиями}.
Неформально говоря, требуется делать следующее: добавлять элемент <<по чуть-чуть>>
так, чтобы он не был виден раньше времени, а потом атомарно <<переключиться>>,
чтобы добавленный элемент появился в куче. Кроме того, необходима возможность
прервать добавление в любой момент.

\begin{definition}
\emph{Куча с версиями} "--- надстройка над двоичной кучей,
поддерживающая следующие операции:
\begin{enumerate}
\item Удалить корень и вернуть два его поддерева. Если в данный момент происходит
    вставка, отменить её и также вернуть вставляемый элемент. После этой операции
    добавлений больше не будет.
\item Если в данный момент не происходит вставка, сделать элемент $x$
    текущим вставляемым элементом.
\item Если в данный момент происходит вставка, проделать $t$ операций по вставке.
\item Если вставка происходила и уже закончена, атомарно добавить вставленный элемент.
\end{enumerate}
\end{definition}

В каждой вершине двоичной кучи хранится два указателя на потомков "--- правого
и левого (возможно, пустые). Будем вместо каждого указателя хранить кортеж
пар (указатель, номер версии), где версия "--- некоторое натуральное число.
Кроме того, в корне делева будет отдельно храниться актуальный номер версии $V$,
изначально равный $0$.
Для того, чтобы получить явное дерево, нужно для каждой вершины взять
указатель с максимальной версией, не превосходящей $V$.

Для выполнения отложенной вставки нужно вставлять элемент как обычно, но вместо
изменения указателей создавать новые, версии $V+1$. После завершения вставки
можно атомарно перейти на новую версию, увеличив $V$. Если необходимо отменить
вставку и удалить корень, нужно просто вернуть оба поддерева корня (согласно версии
$V$) и удалённый элемент, а обоим детям корня установить версию равной $V$.

При добавлении нового указателя в вершину необходимо удалить из неё все указатели,
кроме актуального на данный момент. Несложно видеть, что при этом из каждой
вершины всегда будет исходить не более двух указателей, причём среди них
всегда будет актуальный.
