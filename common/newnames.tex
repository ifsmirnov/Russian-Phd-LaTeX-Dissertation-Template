% Новые переменные, которые могут использоваться во всём проекте
\newcommand{\authorbibtitle}{Публикации автора по теме диссертации}
\newcommand{\fullbibtitle}{Список литературы} % (ГОСТ Р 7.0.11-2011, 4)

\newcommand{\CH}[1][]{
    \ifthenelse{ \equal{#1}{} }
        {\texttt{CascadeHeap}}
        {\texttt{CascadeHeap[#1]}}\xspace}
\newcommand{\SCH}{\texttt{SimpleCascadeHeap}\xspace}
\newcommand{\myB}{\texttt{B}\xspace}
\newcommand{\MH}[1][]{\ifthenelse{ \equal{#1}{} }
        {\texttt{MH}}
        {\texttt{MH$_{#1}$}}\xspace}
\newcommand{\HH}{\texttt{HH}\xspace}

% Theorems
\newtheorem*{theorem-star}{Теорема}
\newtheorem{theorem}{Теорема}[section]
\newtheorem*{theorem-definition-star}{Теорема-определение}
\newtheorem*{corollary-star}{Следствие}
\newtheorem{corollary}{Следствие}[section]
\newtheorem*{property-star}{Свойство}
\newtheorem*{properties}{Свойства}
\newtheorem{property}{Свойство}
\newtheorem*{lem-star}{Лемма}
\newtheorem{lem}{Лемма}[section]
\newtheorem*{proposition-star}{Предложение}
\newtheorem{proposition}{Предложение}
\newtheorem{stage}{Этап}
\newtheorem*{statement}{Утверждение}
\newtheorem*{usage}{Использование}

\theoremstyle{remark}
\newtheorem*{remark}{Замечание}

\theoremstyle{definition}
\newtheorem{problem}{Задача}
\newtheorem{exercise}{Упражнение}

\theoremstyle{definition}
\newtheorem*{definition-star}{Определение}
\newtheorem{definition}{Определение}[section]
\newtheorem*{designation}{Обозначение}

\theoremstyle{definition}
\newtheorem*{example-star}{Пример}
\newtheorem{example}{Пример}
\newtheorem*{examples}{Примеры}
\newtheorem{case}{Случай}
\newtheorem*{case-star}{Случай}

\newcommand{\myX}{\mathcal{X}}
\newcommand{\myH}{\mathcal{H}}

\DeclareMathOperator{\lev}{Level}
