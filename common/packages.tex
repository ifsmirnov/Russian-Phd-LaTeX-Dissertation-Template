%%% Проверка используемого TeX-движка %%%
\usepackage{iftex}[2013/04/04]
\newif\ifxetexorluatex   % определяем новый условный оператор (http://tex.stackexchange.com/a/47579/79756)
\ifXeTeX
    \xetexorluatextrue
\else
    \ifLuaTeX
        \xetexorluatextrue
    \else
        \xetexorluatexfalse
    \fi
\fi

%%% Поля и разметка страницы %%%
\usepackage{pdflscape}                              % Для включения альбомных страниц
\usepackage{geometry}                               % Для последующего задания полей

%%% Математические пакеты %%%
\usepackage{amsthm,amsfonts,amsmath,amssymb,amscd}  % Математические дополнения от AMS
\usepackage{mathtools}                              % Добавляет окружение multlined

%%%% Установки для размера шрифта 14 pt %%%%
%% Формирование переменных и констант для сравнения (один раз для всех подключаемых файлов)%%
%% должно располагаться до вызова пакета fontspec или polyglossia, потому что они сбивают его работу
\newlength{\curtextsize}
\newlength{\bigtextsize}
\setlength{\bigtextsize}{13.9pt}

\makeatletter
%\show\f@size                                       % неплохо для отслеживания, но вызывает стопорение процесса, если документ компилируется без команды  -interaction=nonstopmode 
\setlength{\curtextsize}{\f@size pt}
\makeatother

%%% Кодировки и шрифты %%%
\ifxetexorluatex
    \usepackage{polyglossia}[2014/05/21]            % Поддержка многоязычности (fontspec подгружается автоматически)
\else
    \RequirePDFTeX                                  % tests for PDFTEX use and throws an error if a different engine is being used
   %%% Решение проблемы копирования текста в буфер кракозябрами
%    \input glyphtounicode.tex
%    \input glyphtounicode-cmr.tex %from pdfx package
%    \pdfgentounicode=1
    \usepackage{cmap}                               % Улучшенный поиск русских слов в полученном pdf-файле
    \defaulthyphenchar=127                          % Если стоит до fontenc, то переносы не впишутся в выделяемый текст при копировании его в буфер обмена
    \usepackage[T2A]{fontenc}                       % Поддержка русских букв
    \usepackage[utf8]{inputenc}[2014/04/30]         % Кодировка utf8
    \usepackage[english, russian]{babel}[2014/03/24]% Языки: русский, английский
    \IfFileExists{pscyr.sty}{\usepackage{pscyr}}{}  % Красивые русские шрифты
\fi

%%% Оформление абзацев %%%
\usepackage{indentfirst}                            % Красная строка

%%% Цвета %%%
\usepackage[dvipsnames,usenames]{color}
\usepackage{colortbl}
%\usepackage[dvipsnames, table, hyperref, cmyk]{xcolor} % Вероятно, более новый вариант, вместо предыдущих двух строк. Конвертация всех цветов в cmyk заложена как удовлетворение возможного требования типографий. Возможно конвертирование и в rgb.

%%% Таблицы %%%
\usepackage{longtable}                              % Длинные таблицы
\usepackage{multirow,makecell,array}                % Улучшенное форматирование таблиц
\usepackage{booktabs}                               % Возможность оформления таблиц в классическом книжном стиле (при правильном использовании не противоречит ГОСТ)

%%% Общее форматирование
\usepackage{soulutf8}                               % Поддержка переносоустойчивых подчёркиваний и зачёркиваний
\usepackage{icomma}                                 % Запятая в десятичных дробях


%%% Гиперссылки %%%
\usepackage{hyperref}[2012/11/06]

%%% Изображения %%%
\usepackage{graphicx}[2014/04/25]                   % Подключаем пакет работы с графикой

%%% Списки %%%
\usepackage{enumitem}

%%% Подписи %%%
\usepackage{caption}[2013/05/02]                    % Для управления подписями (рисунков и таблиц) % Может управлять номерами рисунков и таблиц с caption %Иногда может управлять заголовками в списках рисунков и таблиц
\usepackage{subcaption}[2013/02/03]                 % Работа с подрисунками и подобным

%%% Интервалы %%%
\usepackage[onehalfspacing]{setspace}               % Опция запуска пакета правит не только интервалы в обычном тексте, но и формульные

%%% Счётчики %%%
\usepackage[figure,table]{totalcount}               % Счётчик рисунков и таблиц
\usepackage{totcount}                               % Пакет создания счётчиков на основе последнего номера подсчитываемого элемента (может требовать дважды компилировать документ)
\usepackage{totpages}                               % Счётчик страниц, совместимый с hyperref (ссылается на номер последней страницы). Желательно ставить последним пакетом в преамбуле

%%% Продвинутое управление групповыми ссылками (пока только формулами) %%%
\ifxetexorluatex
    \usepackage{cleveref}                           % cleveref корректно считывает язык из настроек polyglossia
\else
    \usepackage[russian]{cleveref}                  % cleveref имеет сложности со считыванием языка из babel. Такое решение русификации вывода выбрано вместо определения в documentclass из опасности что-то лишнее передать во все остальные пакеты, включая библиографию.
\fi
\creflabelformat{equation}{#2#1#3}                  % Формат по умолчанию ставил круглые скобки вокруг каждого номера ссылки, теперь просто номера ссылок без какого-либо дополнительного оформления

\DeclareUTFcharacter[\UTFencname]{x0404}{\CYRIE}
\DeclareUTFcharacter[\UTFencname]{x0405}{\CYRDZE}
\DeclareUTFcharacter[\UTFencname]{x0406}{\CYRII}
\DeclareUTFcharacter[\UTFencname]{x0407}{\CYRYI}
\DeclareUTFcharacter[\UTFencname]{x0408}{\CYRJE}
\DeclareUTFcharacter[\UTFencname]{x0409}{\CYRLJE}
\DeclareUTFcharacter[\UTFencname]{x040A}{\CYRNJE}
\DeclareUTFcharacter[\UTFencname]{x040B}{\CYRTSHE}
\DeclareUTFcharacter[\UTFencname]{x040E}{\CYRUSHRT}
\DeclareUTFcharacter[\UTFencname]{x040F}{\CYRDZHE}
\DeclareUTFcharacter[\UTFencname]{x0410}{\CYRA}
\DeclareUTFcharacter[\UTFencname]{x0411}{\CYRB}
\DeclareUTFcharacter[\UTFencname]{x0412}{\CYRV}
\DeclareUTFcharacter[\UTFencname]{x0413}{\CYRG}
\DeclareUTFcharacter[\UTFencname]{x0414}{\CYRD}
\DeclareUTFcharacter[\UTFencname]{x0415}{\CYRE}
\DeclareUTFcharacter[\UTFencname]{x0416}{\CYRZH}
\DeclareUTFcharacter[\UTFencname]{x0417}{\CYRZ}
\DeclareUTFcharacter[\UTFencname]{x0418}{\CYRI}
\DeclareUTFcharacter[\UTFencname]{x0419}{\CYRISHRT}
\DeclareUTFcharacter[\UTFencname]{x041A}{\CYRK}
\DeclareUTFcharacter[\UTFencname]{x041B}{\CYRL}
\DeclareUTFcharacter[\UTFencname]{x041C}{\CYRM}
\DeclareUTFcharacter[\UTFencname]{x041D}{\CYRN}
\DeclareUTFcharacter[\UTFencname]{x041E}{\CYRO}
\DeclareUTFcharacter[\UTFencname]{x041F}{\CYRP}
\DeclareUTFcharacter[\UTFencname]{x0420}{\CYRR}
\DeclareUTFcharacter[\UTFencname]{x0421}{\CYRS}
\DeclareUTFcharacter[\UTFencname]{x0422}{\CYRT}
\DeclareUTFcharacter[\UTFencname]{x0423}{\CYRU}
\DeclareUTFcharacter[\UTFencname]{x0424}{\CYRF}
\DeclareUTFcharacter[\UTFencname]{x0425}{\CYRH}
\DeclareUTFcharacter[\UTFencname]{x0426}{\CYRC}
\DeclareUTFcharacter[\UTFencname]{x0427}{\CYRCH}
\DeclareUTFcharacter[\UTFencname]{x0428}{\CYRSH}
\DeclareUTFcharacter[\UTFencname]{x0429}{\CYRSHCH}
\DeclareUTFcharacter[\UTFencname]{x042A}{\CYRHRDSN}
\DeclareUTFcharacter[\UTFencname]{x042B}{\CYRERY}
\DeclareUTFcharacter[\UTFencname]{x042C}{\CYRSFTSN}
\DeclareUTFcharacter[\UTFencname]{x042D}{\CYREREV}
\DeclareUTFcharacter[\UTFencname]{x042E}{\CYRYU}
\DeclareUTFcharacter[\UTFencname]{x042F}{\CYRYA}
\DeclareUTFcharacter[\UTFencname]{x0430}{\cyra}
\DeclareUTFcharacter[\UTFencname]{x0431}{\cyrb}
\DeclareUTFcharacter[\UTFencname]{x0432}{\cyrv}
\DeclareUTFcharacter[\UTFencname]{x0433}{\cyrg}
\DeclareUTFcharacter[\UTFencname]{x0434}{\cyrd}
\DeclareUTFcharacter[\UTFencname]{x0435}{\cyre}
\DeclareUTFcharacter[\UTFencname]{x0436}{\cyrzh}
\DeclareUTFcharacter[\UTFencname]{x0437}{\cyrz}
\DeclareUTFcharacter[\UTFencname]{x0438}{\cyri}
\DeclareUTFcharacter[\UTFencname]{x0439}{\cyrishrt}
\DeclareUTFcharacter[\UTFencname]{x043A}{\cyrk}
\DeclareUTFcharacter[\UTFencname]{x043B}{\cyrl}
\DeclareUTFcharacter[\UTFencname]{x043C}{\cyrm}
\DeclareUTFcharacter[\UTFencname]{x043D}{\cyrn}
\DeclareUTFcharacter[\UTFencname]{x043E}{\cyro}
\DeclareUTFcharacter[\UTFencname]{x043F}{\cyrp}
\DeclareUTFcharacter[\UTFencname]{x0440}{\cyrr}
\DeclareUTFcharacter[\UTFencname]{x0441}{\cyrs}
\DeclareUTFcharacter[\UTFencname]{x0442}{\cyrt}
\DeclareUTFcharacter[\UTFencname]{x0443}{\cyru}
\DeclareUTFcharacter[\UTFencname]{x0444}{\cyrf}
\DeclareUTFcharacter[\UTFencname]{x0445}{\cyrh}
\DeclareUTFcharacter[\UTFencname]{x0446}{\cyrc}
\DeclareUTFcharacter[\UTFencname]{x0447}{\cyrch}
\DeclareUTFcharacter[\UTFencname]{x0448}{\cyrsh}
\DeclareUTFcharacter[\UTFencname]{x0449}{\cyrshch}
\DeclareUTFcharacter[\UTFencname]{x044A}{\cyrhrdsn}
\DeclareUTFcharacter[\UTFencname]{x044B}{\cyrery}
\DeclareUTFcharacter[\UTFencname]{x044C}{\cyrsftsn}
\DeclareUTFcharacter[\UTFencname]{x044D}{\cyrerev}
\DeclareUTFcharacter[\UTFencname]{x044E}{\cyryu}
\DeclareUTFcharacter[\UTFencname]{x044F}{\cyrya}
\DeclareUTFcharacter[\UTFencname]{x0451}{\cyryo}
\DeclareUTFcharacter[\UTFencname]{x0452}{\cyrdje}
\DeclareUTFcharacter[\UTFencname]{x0454}{\cyrie}
\DeclareUTFcharacter[\UTFencname]{x0455}{\cyrdze}
\DeclareUTFcharacter[\UTFencname]{x0456}{\cyrii}
\DeclareUTFcharacter[\UTFencname]{x0457}{\cyryi}
\DeclareUTFcharacter[\UTFencname]{x0458}{\cyrje}
\DeclareUTFcharacter[\UTFencname]{x0459}{\cyrlje}
\DeclareUTFcharacter[\UTFencname]{x045A}{\cyrnje}
\DeclareUTFcharacter[\UTFencname]{x045B}{\cyrtshe}
\DeclareUTFcharacter[\UTFencname]{x045E}{\cyrushrt}
\DeclareUTFcharacter[\UTFencname]{x045F}{\cyrdzhe}
\DeclareUTFcharacter[\UTFencname]{x0490}{\CYRGUP}
\DeclareUTFcharacter[\UTFencname]{x0491}{\cyrgup}
\DeclareUTFcharacter[\UTFencname]{x0492}{\CYRGHCRS}
\DeclareUTFcharacter[\UTFencname]{x0493}{\cyrghcrs}
\DeclareUTFcharacter[\UTFencname]{x0496}{\CYRZHDSC}
\DeclareUTFcharacter[\UTFencname]{x0497}{\cyrzhdsc}
\DeclareUTFcharacter[\UTFencname]{x0498}{\CYRZDSC}
\DeclareUTFcharacter[\UTFencname]{x0499}{\cyrzdsc}
\DeclareUTFcharacter[\UTFencname]{x049A}{\CYRKDSC}
\DeclareUTFcharacter[\UTFencname]{x049B}{\cyrkdsc}
\DeclareUTFcharacter[\UTFencname]{x049C}{\CYRKVCRS}
\DeclareUTFcharacter[\UTFencname]{x049D}{\cyrkvcrs}
\DeclareUTFcharacter[\UTFencname]{x04A0}{\CYRKBEAK}
\DeclareUTFcharacter[\UTFencname]{x04A1}{\cyrkbeak}
\DeclareUTFcharacter[\UTFencname]{x04A2}{\CYRNDSC}
\DeclareUTFcharacter[\UTFencname]{x04A3}{\cyrndsc}
\DeclareUTFcharacter[\UTFencname]{x04A4}{\CYRNG}
\DeclareUTFcharacter[\UTFencname]{x04A5}{\cyrng}
\DeclareUTFcharacter[\UTFencname]{x04AA}{\CYRSDSC}
\DeclareUTFcharacter[\UTFencname]{x04AB}{\cyrsdsc}
\DeclareUTFcharacter[\UTFencname]{x04AE}{\CYRY}
\DeclareUTFcharacter[\UTFencname]{x04AF}{\cyry}
\DeclareUTFcharacter[\UTFencname]{x04B0}{\CYRYHCRS}
\DeclareUTFcharacter[\UTFencname]{x04B1}{\cyryhcrs}
\DeclareUTFcharacter[\UTFencname]{x04B2}{\CYRHDSC}
\DeclareUTFcharacter[\UTFencname]{x04B3}{\cyrhdsc}
\DeclareUTFcharacter[\UTFencname]{x04B6}{\CYRCHRDSC}
\DeclareUTFcharacter[\UTFencname]{x04B7}{\cyrchrdsc}
\DeclareUTFcharacter[\UTFencname]{x04B8}{\CYRCHVCRS}
\DeclareUTFcharacter[\UTFencname]{x04B9}{\cyrchvcrs}
\DeclareUTFcharacter[\UTFencname]{x04BA}{\CYRSHHA}
\DeclareUTFcharacter[\UTFencname]{x04BB}{\cyrshha}
\DeclareUTFcharacter[\UTFencname]{x04C0}{\CYRpalochka}
\DeclareUTFcharacter[\UTFencname]{x04D4}{\CYRAE}
\DeclareUTFcharacter[\UTFencname]{x04D5}{\cyrae}
\DeclareUTFcharacter[\UTFencname]{x04D8}{\CYRSCHWA}
\DeclareUTFcharacter[\UTFencname]{x04D9}{\cyrschwa}

